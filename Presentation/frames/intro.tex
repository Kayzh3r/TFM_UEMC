\begin{frame}{Definición y objetivos}
	\begin{block}{\centering \footnotesize Vector de estado de una aeronave}
		%\centering
		Para definir de forma unívoca el estado de una aeronave se necesita conocer su \textbf{vector de estado}. Formado por:
		\begin{itemize}
			\item Velocidad
			\item Posición
			\item Actitud
		\end{itemize}
	\end{block}
	Tipos de navegación:
	\begin{itemize}
		\item Autónoma \arrowTikz{0} no depende de medidas externas ni comunicación con el exterior. Ejemplo: Sistema de Navegación Inercial (INS)
		\item No autónoma \arrowTikz{0} depende de medidas exteriores y/o comunicación con el exterior. Ejemplo: Sistemas de navegación por satélite (GNSS), radio-ayudas o métodos basados en imágenes, entre otros
	\end{itemize}
\end{frame}