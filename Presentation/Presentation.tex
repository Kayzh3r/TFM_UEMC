\documentclass[10pt]{beamer}
\usetheme[
%%% option passed to the outer theme
%    progressstyle=fixedCircCnt,   % fixedCircCnt, movingCircCnt (moving is deault)
  ]{Feather}
  
% If you want to change the colors of the various elements in the theme, edit and uncomment the following lines

% Change the bar colors:
%\setbeamercolor{Feather}{fg=red!20,bg=red}

% Change the color of the structural elements:
%\setbeamercolor{structure}{fg=red}

% Change the frame title text color:
%\setbeamercolor{frametitle}{fg=blue}

% Change the normal text color background:
%\setbeamercolor{normal text}{fg=black,bg=gray!10}


% INCLUDE PACKAGES
\usepackage[utf8]{inputenc}
\usepackage[spanish]{babel}
\usepackage[T1]{fontenc}
\usepackage{helvet}
\usepackage{pdfpages}
%\usepackage{multimedia}
\usepackage{media9}
\usepackage{subcaption}
%\usepackage{dtklogos}
\usepackage{tikz}
\usetikzlibrary{patterns}
\usetikzlibrary{mindmap,shadows}
\usetikzlibrary{trees}
\usetikzlibrary{decorations.pathreplacing}
\usepackage{comment}
%\usepackage[hidelinks,pdfencoding=auto]{hyperref}
\usepackage{enumerate}
\usepackage{amsmath}
\usepackage{empheq}
\usepackage[most]{tcolorbox}
\usepackage[font=scriptsize,labelfont=bf]{caption}
\usetikzlibrary{spy}
\usetikzlibrary{calc}
\usepackage{pdflscape}
\usepackage{glossaries}

%%%%%%%%% Tables Utilities %%%%%%%%%%%
\usepackage{array}
\usepackage{multirow}
\usepackage{booktabs}
\usepackage{longtable}
\usepackage{tabularx}
\usepackage{multicol}
\usepackage{rotating}
\usepackage{tabularx}
\setlength{\tabcolsep}{10pt}
\newcolumntype{L}[1]{>{\raggedright\let\newline\\\arraybackslash\hspace{0pt}}m{#1}}
\newcolumntype{C}[1]{>{\centering\let\newline\\\arraybackslash\hspace{0pt}}m{#1}}
\newcolumntype{R}[1]{>{\raggedleft\let\newline\\\arraybackslash\hspace{0pt}}m{#1}}
%%%%%%%%%%%%%%%%%%%%%%%%%%%%%%%%%%%%%%


\usetikzlibrary{shapes.geometric}
\usetikzlibrary{arrows.meta,arrows}
% FOOTNOTES
\renewcommand*{\thefootnote}{\arabic{footnote}}

% DEFFINING AND REDEFINING COMMANDS
%% ARROW
\newcommand{\arrowTikz}[1]
{
	\begin{tikzpicture}[rotate=#1]
		\coordinate (initPoint)   at (0,0);
		\coordinate (endingPoint) at (0.5,0);
		\draw [line width=1pt,-{Stealth[length=3pt,width=4pt,inset=0.3pt]}](initPoint)--(endingPoint);
	\end{tikzpicture}
}

%% SECTION AND SUBSECTIONS REFERENCE
\newcommand{\refsec}[1]{section~\ref{#1}}
\newcommand{\refsubsec}[1]{subsecsection~\ref{#1}}

%% SUPER & SUB SCRIPT
\newcommand{\superscript}[1]{\ensuremath{^{\textrm{#1}}}}
\newcommand{\subscript}[1]{\ensuremath{_{\textrm{#1}}}}
%% BOLD ITEM
\newcommand\litem[1]{\item{\bfseries #1\enspace} \\}
%% WRITE A VECTOR IN BOLD
\newcommand{\bvec}[1]{\vec{\mathbf{#1}}}
%% PARTIAL DERIVATIVE
\newcommand{\pdv}[2]{\frac{\partial #1}{\partial #2}}
%% BIGO
\DeclareMathAlphabet{\mathpzc}{OT1}{pzc}{m}{it}
\newcommand{\bigO}[1]{$\mathpzc{O}(#1)$}
%% FOR TABLES
\newfloatcommand{capbtabbox}{table}[][\FBwidth]
%% EARTH SPHERE DRAWING
\newcommand\pgfmathsinandcos[3]{%
	\pgfmathsetmacro#1{sin(#3)}%
	\pgfmathsetmacro#2{cos(#3)}%
}
\newcommand\LongitudePlane[3][current plane]{%
	\pgfmathsinandcos\sinEl\cosEl{#2} % elevation
	\pgfmathsinandcos\sint\cost{#3} % azimuth
	\tikzset{#1/.style={cm={\cost,\sint*\sinEl,0,\cosEl,(0,0)}}}
}
\newcommand\LatitudePlane[3][current plane]{%
	\pgfmathsinandcos\sinEl\cosEl{#2} % elevation
	\pgfmathsinandcos\sint\cost{#3} % latitude
	\pgfmathsetmacro\yshift{\cosEl*\sint}
	\tikzset{#1/.style={cm={\cost,0,0,\cost*\sinEl,(0,\yshift)}}} %
}
\newcommand\DrawLongitudeCircle[2][1]{
	\LongitudePlane{\angEl}{#2}
	\tikzset{current plane/.prefix style={scale=#1}}
	% angle of "visibility"
	\pgfmathsetmacro\angVis{atan(sin(#2)*cos(\angEl)/sin(\angEl))} %
	\draw[current plane] (\angVis:1) arc (\angVis:\angVis+180:1);
	\draw[current plane,dashed] (\angVis-180:1) arc (\angVis-180:\angVis:1);
}
\newcommand\DrawLongitudeCircleRed[2][1]{
	\LongitudePlane{\angEl}{#2}
	\tikzset{current plane/.prefix style={scale=#1}}
	% angle of "visibility"
	\pgfmathsetmacro\angVis{atan(sin(#2)*cos(\angEl)/sin(\angEl))} %
	\draw[current plane, color = red] (\angVis:1) arc (\angVis:\angVis+180:1);
	\draw[current plane,dashed, color = red] (\angVis-180:1) arc (\angVis-180:\angVis:1);
}
\newcommand\DrawLatitudeCircle[2][2]{
	\LatitudePlane{\angEl}{#2}
	\tikzset{current plane/.prefix style={scale=#1}}
	\pgfmathsetmacro\sinVis{sin(#2)/cos(#2)*sin(\angEl)/cos(\angEl)}
	% angle of "visibility"
	\pgfmathsetmacro\angVis{asin(min(1,max(\sinVis,-1)))}
	\draw[current plane] (\angVis:1) arc (\angVis:-\angVis-180:1);
	\draw[current plane,dashed] (180-\angVis:1) arc (180-\angVis:\angVis:1);
}
\newcommand\DrawLatitudeCircleRed[2][2]{
	\LatitudePlane{\angEl}{#2}
	\tikzset{current plane/.prefix style={scale=#1}}
	\pgfmathsetmacro\sinVis{sin(#2)/cos(#2)*sin(\angEl)/cos(\angEl)}
	% angle of "visibility"
	\pgfmathsetmacro\angVis{asin(min(1,max(\sinVis,-1)))}
	\draw[current plane,red] (\angVis:1) arc (\angVis:-\angVis-180:1);
	\draw[current plane,dashed,red] (180-\angVis:1) arc (180-\angVis:\angVis:1);
}

%% CAPTION FOR EQUATION SET
\newcounter{equationset}
\newcommand{\equationset}[1]{% \equationset{<caption>}
	\refstepcounter{equationset}% Step counter
	\noindent\makebox[\linewidth]{Ecuaci\'on~\theequationset: #1}}% Print caption

% INFORMATION IN THE TITLE PAGE

\title[] % [] is optional - is placed on the bottom of the sidebar on every slide
{ % is placed on the title page
      \textbf{Eliminación de ruido espectral basado en redes neuronales}
}

\subtitle[Escuela Politécnica Superior]
{
      \textbf{Escuela Politécnica Superior}
}

\author[Ignazio F.Finazzi]
{      Ignazio F.Finazzi \\
      {}
}

\institute[]
{
      Escuela Politécnica Superior\\
      Universidad Europea Miguel de Cervantes\\
  
  %there must be an empty line above this line - otherwise some unwanted space is added between the university and the country (I do not know why;( )
}

\date{\today}


% GLOSSARY
%% Definitions and acronyms entries

% \newacronym{uav}{UAV}{Unmanned Aerial Vehicle}
% Reference singular: \gls{uav}   --> UAV
% Reference plural:   \glspl{uav} --> UAVs

% The following definitions will go in the main glossary

% Basics
\newacronym{TFM}{TFM}{Trabajo de Fin de Máster}
\newacronym{UEMC}{UEMC}{Universidad Europea Miguel de Cervantes}
\newacronym{TTF}{TTF}{TrueType Font}
\newacronym{OTF}{OTF}{OpenType Font}


%-------------------------------------------------------
% THE BODY OF THE PRESENTATION
%-------------------------------------------------------

\begin{document}
	% COVER-PAGE
	\bgroup
%\setbeamercolor{background canvas}{bg=beamer@headercolor}
\usebackgroundtemplate{}
\begin{frame}[plain,noframenumbering]{}
	\begin{minipage}[t]{\linewidth-2\fboxsep-2\fboxrule}
		\centering
		\vspace{-0.015\paperheight}
		\hspace*{-1.1385\paperwidth}
		\tikzGraphic
	\end{minipage}
	\hspace*{-1.15\SidebarWidth}
	\centering
	\huge
	\begin{minipage}[c][\textheight][c]{\textwidth}
		
		\centering
		
		{\usebeamerfont{institute}\usebeamercolor[bg]{title}\insertinstitute}\vspace*{30pt}
		
		{\usebeamerfont{title}\usebeamercolor[bg]{title}\inserttitle}\vspace*{30pt}
		
		%{\usebeamerfont{subtitle}\usebeamercolor[bg]{subtitle}\insertsubtitle}\vspace*{30pt}
		
		{\usebeamerfont{author}\usebeamercolor[bg]{title}Autor:\insertauthor}\vspace*{1pt}
		
		{\usebeamerfont{author}\usebeamercolor[bg]{title}Tutora: Patricia Jiménez Fernández}\vspace*{30pt}
		
		{\usebeamerfont{date}\usebeamercolor[bg]{title}\insertdate}\vspace*{\baselineskip}
		
	\end{minipage}
\end{frame}
\egroup
	% TABLE OF CONTENTS
	\begin{frame}{Índice}{}
	\scriptsize{
	\tableofcontents}
\end{frame}
	\scriptsize
	% INTRODUCCION
	\section{Introducción}
		\subsection{Definición y objetivos}
			\begin{frame}{Definición y objetivos}
	\begin{block}{\centering \footnotesize Definición y objetivos}
		%\centering
		El desarrollo del trabajo propuesto consiste en el diseño y prototipado de un sistema la eliminación de ruido en conversaciones habladas.
		\begin{itemize}
			\item Ejecución en tiempo real
			\item Eliminación de gran abanico de ruidos
		\end{itemize}
	\end{block}
\end{frame}
		\subsection{Introducción al Procesamiento de señal}
		\subsection{Redes LSTM}
		\subsection{Estado del arte}
	% DATA GATHERING
	\section{Obtención, procesado y almacenamiento de datos}
		\subsection{Datos de Audio}
		\subsection{Datos de Ruido}
	% EDA
	\section{Análisis exploratorio de datos}
		\subsection{Análisis de integridad}
		\subsection{Análisis de los datos de audio}
	% MODEL DESIGN
	\section{Diseño e implementación de modelos}
		\subsection{Pre-procesado de datos}
		\subsection{Modelo de capas LSTM}
	% RESULTS
	\section{Análisis de los resultados obtenidos}
	% CONCLUSIONS
	\section{Resultados y conclusiones}
	
	% THANKS PAGE
	\bgroup
%\setbeamercolor{background canvas}{bg=beamer@headercolor}
\usebackgroundtemplate{}
\begin{frame}[plain,noframenumbering]{}
	\begin{minipage}[t]{\linewidth-2\fboxsep-2\fboxrule}
		\centering
		\vspace{-0.329\paperheight}
		\hspace*{-1.133\paperwidth}
		\tikzGraphic
	\end{minipage}
	\hspace*{-1.15\SidebarWidth}
	\centering
	\huge
	\ \textcolor{white}{GRACIAS POR SU}
	%\vspace{1 cm}
	\\
	\hspace*{-1.15\SidebarWidth} 
	\textcolor{white}{ATENCIÓN}
	\\
	\vspace*{1 cm}
	\hspace*{-1.15\SidebarWidth}
	%\textcolor{white}{ifinazzi@gte.esi.us.es}
	\vspace*{-2 cm}
\end{frame}
\egroup
\end{document}