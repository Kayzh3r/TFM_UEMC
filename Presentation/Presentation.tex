\documentclass[10pt]{beamer}
\usetheme[
%%% option passed to the outer theme
%    progressstyle=fixedCircCnt,   % fixedCircCnt, movingCircCnt (moving is deault)
  ]{Feather}
  
% If you want to change the colors of the various elements in the theme, edit and uncomment the following lines

% Change the bar colors:
%\setbeamercolor{Feather}{fg=red!20,bg=red}

% Change the color of the structural elements:
%\setbeamercolor{structure}{fg=red}

% Change the frame title text color:
%\setbeamercolor{frametitle}{fg=blue}

% Change the normal text color background:
%\setbeamercolor{normal text}{fg=black,bg=gray!10}


% INCLUDE PACKAGES
\usepackage[utf8]{inputenc}
\usepackage[spanish]{babel}
\usepackage[T1]{fontenc}
\usepackage{helvet}
\usepackage{pdfpages}
%\usepackage{multimedia}
\usepackage{media9}
\usepackage{subcaption}
%\usepackage{dtklogos}
\usepackage{tikz}
\usetikzlibrary{patterns}
\usetikzlibrary{mindmap,shadows}
\usetikzlibrary{trees}
\usetikzlibrary{decorations.pathreplacing}
\usepackage{comment}
%\usepackage[hidelinks,pdfencoding=auto]{hyperref}
\usepackage{enumerate}
\usepackage{amsmath}
\usepackage{empheq}
\usepackage[most]{tcolorbox}
\usepackage[font=scriptsize,labelfont=bf]{caption}
\usetikzlibrary{spy}
\usetikzlibrary{calc}
\usepackage{pdflscape}
\usepackage{glossaries}

%%%%%%%%% Tables Utilities %%%%%%%%%%%
\usepackage{array}
\usepackage{multirow}
\usepackage{booktabs}
\usepackage{longtable}
\usepackage{tabularx}
\usepackage{multicol}
\usepackage{rotating}
\usepackage{tabularx}
\setlength{\tabcolsep}{10pt}
\newcolumntype{L}[1]{>{\raggedright\let\newline\\\arraybackslash\hspace{0pt}}m{#1}}
\newcolumntype{C}[1]{>{\centering\let\newline\\\arraybackslash\hspace{0pt}}m{#1}}
\newcolumntype{R}[1]{>{\raggedleft\let\newline\\\arraybackslash\hspace{0pt}}m{#1}}
%%%%%%%%%%%%%%%%%%%%%%%%%%%%%%%%%%%%%%


\usetikzlibrary{shapes.geometric}
\usetikzlibrary{arrows.meta,arrows}
% FOOTNOTES
\renewcommand*{\thefootnote}{\arabic{footnote}}

% DEFFINING AND REDEFINING COMMANDS
% ARROW
\newcommand{\arrowTikz}[1]
{
	\begin{tikzpicture}[rotate=#1]
		\coordinate (initPoint)   at (0,0);
		\coordinate (endingPoint) at (0.5,0);
		\draw [line width=1pt,-{Stealth[length=3pt,width=4pt,inset=0.3pt]}](initPoint)--(endingPoint);
	\end{tikzpicture}
}
\newcommand{\superscript}[1]{\ensuremath{^{\textrm{#1}}}}
\newcommand{\subscript}[1]{\ensuremath{_{\textrm{#1}}}}

% COLORED HYPERLINKS
\newcommand{\chref}[2]{
	\href{#1}{{\usebeamercolor[bg]{Feather}#2}}
}
% GREEN CHECK
\newcommand{\checkTikz}[1]
{
	\begin{tikzpicture}[scale=#1]
		\tikzstyle{invisble} = [outer sep=0,inner sep=0,minimum size=0]
		\draw [fill=green,color=green](-1,0) node [invisble] (v1) {} -- (-0.5,-0.5) node [invisble] {} -- (1,1) node [invisble] {} -- (-0.5,-1) node [invisble] {} -- (v1);
	\end{tikzpicture}
}
% RED CROSS
\newcommand{\redcrossTikz}[1]
{
	\begin{tikzpicture}[scale=#1]
		\tikzstyle{invisble} = [outer sep=0,inner sep=0,minimum size=0]
		\draw [fill=red,color=red]
		(0,0) node [invisble] (v1) {} -- 
		(0.25,0) node [invisble] {} -- 
		(0.75,-0.5) node [invisble] {} -- 
		(1.25,0) node [invisble] {} -- 
		(1.5,0) node [invisble] {} -- 
		(1.5,-0.25) node [invisble] {} -- 
		(1,-0.75) node [invisble] {} -- 
		(1.5,-1.25) node [invisble] {} -- 
		(1.5,-1.5) node [invisble] {} -- 
		(1.25,-1.5) node [invisble] {} -- 
		(0.75,-1) node [invisble] {} -- 
		(0.25,-1.5) node [invisble] {} -- 
		(0,-1.5) node [invisble] {} --
		(0,-1.25) node [invisble] {} -- 
		(0.5,-0.75) node [invisble] {} -- 
		(0,-0.25) node [invisble] {} -- (v1);
	\end{tikzpicture}
}
% BALLS SCHEMA
\newcommand*{\info}[4][16.3]{%
	\node [ annotation, #3, scale=0.65, text width = #1em,
	inner sep = 2mm ] at (#2) {%
		\list{$\bullet$}{\topsep=0pt\itemsep=0pt\parsep=0pt
			\parskip=0pt\labelwidth=8pt\leftmargin=8pt
			\itemindent=0pt\labelsep=2pt}%
		#4
		\endlist
	};
}

% INFORMATION IN THE TITLE PAGE

\title[] % [] is optional - is placed on the bottom of the sidebar on every slide
{ % is placed on the title page
      \textbf{Eliminación de ruido espectral basado en redes neuronales}
}

\subtitle[Escuela Politécnica Superior]
{
      \textbf{Escuela Politécnica Superior}
}

\author[Ignazio F.Finazzi]
{      Ignazio F.Finazzi \\
      {}
}

\institute[]
{
      Escuela Politécnica Superior\\
      Universidad Europea Miguel de Cervantes\\
  
  %there must be an empty line above this line - otherwise some unwanted space is added between the university and the country (I do not know why;( )
}

\date{\today}


% GLOSSARY
%% Definitions and acronyms entries

% \newacronym{uav}{UAV}{Unmanned Aerial Vehicle}
% Reference singular: \gls{uav}   --> UAV
% Reference plural:   \glspl{uav} --> UAVs

% The following definitions will go in the main glossary

% Basics
\newacronym{TFM}{TFM}{Trabajo de Fin de Máster}
\newacronym{UEMC}{UEMC}{Universidad Europea Miguel de Cervantes}
\newacronym{TTF}{TTF}{TrueType Font}
\newacronym{OTF}{OTF}{OpenType Font}
\newacronym{WWW}{WWW}{World Wide Web}
\newacronym{HTTP}{HTTP}{HyperText Transfer Protocol}
\newglossaryentry{framework}{name={Framework},description={Un framework, entorno de trabajo​ o marco de trabajo​ es un conjunto estandarizado de conceptos, prácticas y criterios para enfocar un tipo de problemática particular que sirve como referencia, para enfrentar y resolver nuevos problemas de índole similar.}}
\newglossaryentry{scraper}{name={Web scraping},description={Es una técnica utilizada mediante programas de software para extraer información de sitios web.1​ Usualmente, estos programas simulan la navegación de un humano en la \gls{WWW} ya sea utilizando el protocolo \gls{HTTP} manualmente, o incrustando un navegador en una aplicación.}}
\newacronym{API}{API}{Application Programming Interface}
\newacronym{HTML}{HTML}{HyperText Markup Language}
\newacronym{JSON}{JSON}{JavaScript Object Notation}
\newacronym{URL}{URL}{Uniform Resource Locator}
\newglossaryentry{GET}{name={GET},description={Es un método de petición del protocolo \gls{HTTP} que se utiliza para solicitar datos a través unos filtros definidos en la \gls{URL} con la que se hace la petición.}}
\newglossaryentry{HDF5}{name={HDF5},description={\gls{HDF} versión 5. Es un formato binario organizado utilizado para almacenar y acceder a grandes volúmenes de datos de manera indexada. Está estructurado de manera similar a una carpeta, soporta links virtuales para referenciar datasets, valores de relleno por defecto y metadatos, entre otras capacidades.}}
\newacronym{HDF}{HDF}{Hierarchical Data Format}

\newacronym{ADC}{ADC}{Analogical to Digital Converter}
\newacronym{DSP}{DSP}{Digital Signal Procesing}
\newacronym{ASCII}{ASCII}{American Standard Code for Information Interchange}

%% MACHINE LEARNING
\newglossaryentry{dataset}{name={Dataset},description={Conjunto de datos, normalmente estructurados.}}
\newacronym{ML}{ML}{Machine Learning}
\newacronym{LSTM}{LSTM}{Long Short-Term Memory}
\newacronym{GRU}{GRU}{Gated Recurrent Unit}

%% OTHER
\newacronym{GPS}{GPS}{Global Positioning System}

%% DSP
\newacronym{FFT}{FFT}{Fast Fourier Transform}
\newacronym{SNR}{SNR}{Signal to Noise Ratio}
\newacronym{STFT}{STFT}{Short-Time Fourier Transform}

%-------------------------------------------------------
% THE BODY OF THE PRESENTATION
%-------------------------------------------------------

\begin{document}
	% COVER-PAGE
	\bgroup
%\setbeamercolor{background canvas}{bg=beamer@headercolor}
\usebackgroundtemplate{}
\begin{frame}[plain,noframenumbering]{}
	\begin{minipage}[t]{\linewidth-2\fboxsep-2\fboxrule}
		\centering
		\vspace{-0.015\paperheight}
		\hspace*{-1.1385\paperwidth}
		\tikzGraphic
	\end{minipage}
	\hspace*{-1.15\SidebarWidth}
	\centering
	\huge
	\begin{minipage}[c][\textheight][c]{\textwidth}
		
		\centering
		
		{\usebeamerfont{institute}\usebeamercolor[bg]{title}\insertinstitute}\vspace*{30pt}
		
		{\usebeamerfont{title}\usebeamercolor[bg]{title}\inserttitle}\vspace*{30pt}
		
		%{\usebeamerfont{subtitle}\usebeamercolor[bg]{subtitle}\insertsubtitle}\vspace*{30pt}
		
		{\usebeamerfont{author}\usebeamercolor[bg]{title}Autor:\insertauthor}\vspace*{1pt}
		
		{\usebeamerfont{author}\usebeamercolor[bg]{title}Tutora: Patricia Jiménez Fernández}\vspace*{30pt}
		
		{\usebeamerfont{date}\usebeamercolor[bg]{title}\insertdate}\vspace*{\baselineskip}
		
	\end{minipage}
\end{frame}
\egroup
	% TABLE OF CONTENTS
	\begin{frame}{Índice}{}
	\scriptsize{
	\tableofcontents}
\end{frame}
	\scriptsize
	% INTRODUCCION
	\section{Introducción}
		\subsection{Definición y objetivos}
			\begin{frame}{Definición y objetivos}
	\begin{block}{\centering \footnotesize Definición y objetivos}
		%\centering
		El desarrollo del trabajo propuesto consiste en el diseño y prototipado de un sistema la eliminación de ruido en conversaciones habladas.
		\begin{itemize}
			\item Velocidad
			\item Posición
			\item Actitud
		\end{itemize}
	\end{block}
	Tipos de navegación:
	\begin{itemize}
		\item Autónoma \arrowTikz{0} no depende de medidas externas ni comunicación con el exterior. Ejemplo: Sistema de Navegación Inercial (INS)
		\item No autónoma \arrowTikz{0} depende de medidas exteriores y/o comunicación con el exterior. Ejemplo: Sistemas de navegación por satélite (GNSS), radio-ayudas o métodos basados en imágenes, entre otros
	\end{itemize}
\end{frame}
		\subsection{Introducción al Procesamiento de señal}
		\subsection{Redes LSTM}
		\subsection{Estado del arte}
	% DATA GATHERING
	\section{Obtención, procesado y almacenamiento de datos}
		\subsection{Datos de Audio}
		\subsection{Datos de Ruido}
	% EDA
	\section{Análisis exploratorio de datos}
		\subsection{Análisis de integridad}
		\subsection{Análisis de los datos de audio}
	% MODEL DESIGN
	\section{Diseño e implementación de modelos}
		\subsection{Pre-procesado de datos}
		\subsection{Modelo de capas LSTM}
	% RESULTS
	\section{Análisis de los resultados obtenidos}
	% CONCLUSIONS
	\section{Resultados y conclusiones}
	
	% THANKS PAGE
	\bgroup
%\setbeamercolor{background canvas}{bg=beamer@headercolor}
\usebackgroundtemplate{}
\begin{frame}[plain,noframenumbering]{}
	\begin{minipage}[t]{\linewidth-2\fboxsep-2\fboxrule}
		\centering
		\vspace{-0.329\paperheight}
		\hspace*{-1.133\paperwidth}
		\tikzGraphic
	\end{minipage}
	\hspace*{-1.15\SidebarWidth}
	\centering
	\huge
	\ \textcolor{white}{GRACIAS POR SU}
	%\vspace{1 cm}
	\\
	\hspace*{-1.15\SidebarWidth} 
	\textcolor{white}{ATENCIÓN}
	\\
	\vspace*{1 cm}
	\hspace*{-1.15\SidebarWidth}
	%\textcolor{white}{ifinazzi@gte.esi.us.es}
	\vspace*{-2 cm}
\end{frame}
\egroup
\end{document}