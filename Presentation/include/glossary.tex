% Entradas para definiciones y siglas

% Referenciar una definición o siglas: \gls{uav}
% referenciar en plural: \glspl{uav} --> Resultará en UAVs

% Nota: para poner las tildes, usar por ejemplo informaci{\'o}n
% (ya que el paquete de glosario no admite tildes)

% The following definitions will go in the main glossary

%Various
\newacronym{eu}{EU}{European Union}
\newacronym{esa}{ESA}{European Space Agency}
\newacronym{wgs84}{WGS84}{World Geodetic System 1984}
\newacronym{fpga}{FPGA}{Field Programmable Gate Array}
\newacronym{imu}{IMU}{Inertial Measurement Unit}
\newacronym{foc}{FOC}{Full Operational Capability}
\newacronym{sv}{SV}{Space Vehicle}
\newacronym{ins}{INS}{\textit{Inertial Navigation System}}

%
\newglossaryentry{GMSK}{name={GMSK},description={\textbf{Gaussian Minimum Shift Keying} is a form of phase modulation that is used in a number of portable radio and wireless applications. It has advantages in terms of spectral efficiency as well as having an almost constant amplitude which allows for the use of more efficient transmitter power amplifiers, thereby saving on current consumption, a critical issue for battery power equipment.}}
\newglossaryentry{PSK}{name={PSK},description={\textbf{Phase shift Keying} is a form of modulation used particularly for data transmissions. If offers an effective way of transmitting data. By altering the number of different phase states which can be adopted, the data speeds that can be achieved within a given channel can be increased, but at the cost of lower resilience to noise an interference.}}
\newglossaryentry{mQAM}{name={mQAM},description={\textbf{Quadrature Amplitude Modulation} is a method of combining two amplitude-modulated signals into a single channel, thereby doubling the effective bandwidth. QAM is used with pulse amplitude modulation in digital systems, especially in wireless applications.}}
\newglossaryentry{ARFCN}{name={ARFCN},description={\textbf{Absolute Radio-Frequency Channel Number} is a code that specifies a pair of physical radio carriers used for transmission and reception in a land mobile radio system, one for the uplink signal and one for the downlink signal.}}

% The following definitions will go in the list of acronyms
\newacronym{uav}{UAV}{\textit{Unmanned Aircraft Vehicle}}

% POSITONING METHODS
\newacronym{CoO}{CoO}{Cell of Origin}
\newacronym{FP}{FP}{Fingerprinting}
\newacronym{RSSI}{RSSI}{Received Signal Strength Indication}
\newacronym{LoS}{LoS}{Line of Sight}
\newacronym{NLoS}{NLoS}{Non-Line of Sight}
\newacronym{SNR}{SNR}{Signal-to-Noise Ratio}
\newacronym{OTDoA}{OTDoA}{Observed Time Difference of Arrival}

% VOR acronyms 
\newacronym{VOR}{VOR}{\textit{Very High Frequency Omnidirectional Range}}
\newacronym{CVOR}{CVOR}{Conventional VOR}
\newacronym{DVOR}{DVOR}{Doppler VOR}
\newacronym{VHF}{VHF}{Very High frequency}
\newacronym{UHF}{UHF}{Ultra High frequency}

% ADS-B
\newacronym{ADS-B}{ADS-B}{Automatic Dependent Surveillance Broadcast}
\newacronym{PPM}{PPM}{Pulse Position Modulation}

% DME
\newacronym{DME}{DME}{\textit{Distance Measuring Equipment}}
\newacronym{DME-P}{DME-P}{Distance Measuring Equipment Precision}

% ADF-NDB
\newacronym{ADF}{ADF}{Automatic Direction Finder}
\newacronym{NDB}{NDB}{\textit{Non Directional Beacon}}
\newacronym{DF}{DF}{Direction Finder}

% ILS
\newacronym{ILS}{ILS}{\textit{Instrumental Landing System}}
\newacronym{SBO}{SBO}{Side Band Only}
\newacronym{CSB}{CSB}{Carrier plus Side Band}
\newacronym{OM}{OM}{Outer Maker}
\newacronym{MM}{MM}{Middle Marker}
\newacronym{IM}{IM}{Inner Marker}
\newacronym{DDM}{DDM}{Depth of Modulation}
\newacronym{RVR}{RVR}{Runway Visual Range}
\newacronym{LLZ}{LLZ}{Localizer}
\newacronym{GP}{GP}{Glide Path}

% MLS
\newacronym{MLS}{MLS}{Microwave Landing System}

% DVB-T
\newacronym{DVB}{DVB}{Digital Video Broadcasting}
\newacronym{DVB-T}{DVB-T}{Digital Video Broadcasting-Terrestrial}
\newacronym{OFDM}{OFDM}{Orthogonal frequency-division multiplexing}
\newacronym{SFN}{SFN}{Single Frequency Network}
\newacronym{DVB-SH}{DVB-SH}{Digital Video Broadcasting Satellite services to Handhelds}

% CELLULAR NETWORKS
\newacronym{UMTS}{UMTS}{Universal Mobile Telecommunications System}
\newacronym{GSM}{GSM}{Global System for Mobile communication}
\newacronym{LTE}{LTE}{Long Term Evolution}
\newacronym{BS}{BS}{Base Station}
\newacronym{MS}{MS}{Mobile Station}
\newacronym{SCS}{SCS}{Stereo Channel Separation}
\newacronym{A-FLT}{A-FLT}{Advanced Forward Link Trilateration}
\newacronym{ECoO}{ECoO}{Enhanced Cell of Origin}
\newacronym{TDMA}{TDMA}{Time Division Multiple Access}
\newacronym{W-CDMA}{W-CDMA}{Wideband-Code Division Multiple Access}
\newacronym{RDS}{RDS}{Radio Data System}

% GSM
\newacronym{BCH}{BCH}{Broadcasting Channel}
\newacronym{BSIC}{BSIC}{Base Station Identity Code}
\newacronym{RACH}{RACH}{Random-Access Channel}
\newacronym{AGCH}{AGCH}{Access Grant Channel}
% HARDWARE
\newacronym{ADC}{ADC}{\textit{Analog-to-Digital Converter}}
\newacronym{DAC}{DAC}{Digital-to-Analog Converter}
\newacronym{FPGA}{FPGA}{Field Programmable Gate Array}
\newacronym{SDR}{SDR}{Software Defined Radio}
\newacronym{DDC}{DDC}{Digital Down Converter}
\newacronym{DUC}{DUC}{Digital Up Converter}
\newacronym{OEM}{OEM}{Original Equipment Manufacturer}
\newacronym{COTS}{COTS}{Commercial Off-The-Shelf}
\newacronym{RF}{RF}{Radio Frequency}
\newacronym{IF}{IF}{Intermediate Frequency}
\newacronym{OS}{OS}{Operative System}
\newacronym{GRC}{GRC}{GNU Radio Companion}
\newacronym{USRP}{USRP}{Universal Software Radio Peripheral}
\newacronym{LO}{LO}{Local Oscillator}
\newacronym{ARM}{ARM}{Acorn \gls{RISC} Machine}
\newacronym{RISC}{RISC}{Reduced Instruction Set Computing}
\newacronym{DSP}{DSP}{Digital Signal Processor}
\newacronym{VHDL}{VHDL}{\gls{VHSIC} \gls{HDL}}
\newacronym{VHSIC}{VHSIC}{Very High Speed Integrated Circuit}
\newacronym{HDL}{HDL}{Hardware Description Language}

% POSITIONING METHODS
\newacronym{ToA}{ToA}{Time of Arrival}
\newacronym{TDoA}{TDoA}{Time Difference of Arrival}
\newacronym{RTT}{RTT}{Round Trip Time}
\newacronym{TWR}{TWR}{Two Way Ranging}
\newacronym{PoA}{PoA}{Phase of Arrival}
\newacronym{NFER}{NFER}{Near-Field Electromagnetic Ranging}
\newacronym{AoA}{AoA}{Angle of Arrival}

% INS
\newacronym{INS}{INS}{\textit{Inertial Navigation System}}
\newacronym{FEN}{FEN}{\textit{Fundamental Equation of Navigation}}
\newacronym{ECI}{ECI}{\textit{Earth-centred Inertial}}
\newacronym{LTP}{LTP}{Local tangent plane}
\newacronym{ENU}{ENU}{East-North-Up}
\newacronym{NED}{NED}{North East Down}
% GNSS
\newacronym{GNSS}{GNSS}{\textit{Global Navigation Satellite System}}
\newacronym{NAVSAT}{NAVSAT}{Navy Navigation Satellite System}
\newacronym{SSBN}{SSBN}{Nuclear-powered ballistic missile submarines}
\newacronym{APL}{APL}{Applied Physics Laboratory}
\newacronym{GPS}{GPS}{Global Positioning System}
\newacronym{DoD}{DoD}{Department of Defense}
\newacronym{GLONASS}{GLONASS}{GLObal NAvigation Satellite System}
\newacronym{UTC}{UTC}{Universal Time Coordinated}
\newacronym{TTC}{TT\&C}{Telemetry Tracking and Control}
\newacronym{MEO}{MEO}{Medium Earth Orbit}
\newacronym{GEO}{GEO}{Geostationary Earth Orbit}
\newacronym{HEO}{HEO}{Highly Elliptical Orbit}
\newacronym{GSO}{GSO}{Geosynchronous Satellite Orbit}
\newacronym{IGSO}{IGSO}{Inclined Geosynchronous Satellite Orbit}
\newacronym{CDMA}{CDMA}{Code Division Multiple Access}
\newacronym{QPSK}{QPSK}{Quadrature Phase Shift Keying}
\newacronym{BDT}{BDT}{BeiDou Time}
\newacronym{CAST}{CAST}{China Academy of Space Technology}
\newacronym{CNSS}{CNSS}{Compass Navigation Satellite System}
\newacronym{RHCP}{RHCP}{Right-Hand Circularly Polarized}
\newacronym{JAXA}{JAXA}{Japan Aerospace Exploration Agency}
\newacronym{QZS}{QZS}{Quasi-Zenit Satellite}
\newacronym{TCS}{TCS}{Traking Control Stations}
\newacronym{URA}{URA}{User Range Accuracy}
\newacronym{UDRA}{UDRA}{User Differential Range Accuracy}
\newacronym{NMCT}{NMCT}{Navigation Message Correction Table}
\newacronym{EDC}{EDC}{Ephemeris Differential Corrections}
\newacronym{CDC}{CDC}{Clock Differential Corrections}
\newacronym{NSC}{NSC}{Non-Standard Code}
\newacronym{JGS}{JGS}{Japan satellite navigation Geodetic System}
\newacronym{ITRS}{ITRS}{International Terrestrial Reference System}
\newglossaryentry{UERE}{%
	name={UERE},%
	description={User Equivalent Rate Error. It is a statistical ranging error that represents the total effect of all the contributing error sources. UERE is defined as the root-sum-square of the various error sources affecting the pseudorange measurement, all expressed in units of length. It is calculated:
		\begin{equation}
			\sigma_{UERE}=\sqrt{\sigma_{R1}^{2}+\sigma_{R2}^{2}+\dots+\sigma_{Rn}^{2}}\nonumber
		\end{equation}
		where $\sigma_{R1}$,$\sigma_{R2}$,$\dots$,$\sigma_{Rn}$ are the root mean square range errors due to various error sources\cite{Sunehra2013}}%
	\newacronym{UERE}{UERE}{User Equivalent Rate Error}%
}
\newacronym{CGCS2000}{CGCS2000}{Chinese Geodetic Coordinate System 2000}
\newacronym{CGS}{CGS}{Compass Geodetic System}

% AUGMENTATION SYSTEMS
% SPACE-BASED
\newacronym{SBAS}{SBAS}{Space-based Augmentation System}
\newacronym{WAD}{WAD}{Wide Area Differential}
\newacronym{UDRE}{UDRE}{User Differential Range Error}
\newacronym{GIVE}{GIVE}{Grid Ionospheric Vertical Error}
\newacronym{DGPS}{DGPS}{Differential GPS}
	% WAAS
	\newacronym{WAAS}{WAAS}{Wide-Area Augmentation System}
	\newacronym{WRS}{WRS}{Wide-Area Reference Station}
	\newacronym{WMS}{WMS}{Wide-Area Master Station}
	\newacronym{GES}{GES}{Groun Earth Station}
	% MSAS
	\newacronym{MSAS}{MSAS}{Multi-functional Satellite Augmentation System}
	\newacronym{MTSAT}{MTSAT}{Multifunctional Transport SATellites}
	\newacronym{QZSS}{QZSS}{Quasi-Zenit Satellite System}
	% EGNOS
	\newacronym{EGNOS}{EGNOS}{European Geoestationary Navigation Overlay System}
	\newacronym{RIMS}{RIMS}{Receiver Integrity Monitoring Stations}
	\newacronym{EWAN}{EWAN}{EGNOS Wide Area Network}
	\newacronym{MCC}{MCC}{Master Control Center}
	\newacronym{NLES}{NLES}{Navigation Land Earth Stations}
	% GAGAN
	\newacronym{GAGAN}{GAGAN}{GPS Aided Geo Augmented Navigation}
	% SNAS
	\newacronym{SNAS}{SNAS}{Satellite Navegation Augmentation System}
	% SDCM
	\newacronym{SDCM}{SDCM}{System for Differential Correction and Monitoring}
	% International organizations and reglamentations
	\newacronym{SARP}{SARP}{Standard And Recomended Practices}
	\newacronym{FAA}{FAA}{Federal Aviation Administration}
	\newacronym{ESA}{ESA}{European Space Agency}
	\newacronym{EUROCONTROL}{EUROCONTROL}{European Organization for the Safety of Air Navigation}
% GROUND-BASED
\newacronym{GBAS}{GBAS}{Ground-Based Augmentation System}
\newacronym{GRAS}{GRAS}{Ground-Based Regional Augmentation System}
\newacronym{LAAS}{LAAS}{Local Area Augmentation System}
\newacronym{ICAO}{ICAO}{International Civil Aviation Organization}
\newacronym{PRC}{PRC}{PseudoRange Corrections}
\newacronym{RRC}{RRC}{Rate Range Corrections}

% GPS
\newacronym{MCS}{MCS}{Master Control Station}
\newacronym{CS}{CS}{Control Segment}
\newacronym{SS}{SS}{Space Segment}
%\newacronym{US}{US}{User Segment}
\newacronym{PPS}{PPS}{Precise Positioning Service}
\newacronym{SIS}{SIS}{System-In-Space}
\newacronym{NGA}{NGA}{National Geospatial-Intelligence Agency}
\newacronym{SV}{SV}{Space Vehicle}
\newacronym{CA}{C/A-code}{Coarse-Acquisition code}
\newacronym{Pcode}{P-code}{Precision code}
\newacronym{PRN}{PRN}{Pseudo Random Noise}
\newacronym{PDOP}{PDOP}{Position Dilution Of Precision}
\newacronym{HDOP}{HDOP}{Horizontal Dilution Of Precision}
\newacronym{TDOP}{TDOP}{Time Dilution Of Precision}
\newacronym{VDOP}{VDOP}{Vertical Dilution Of Precision}

%GALILEO
\newacronym{sar}{SAR}{Search \& Rescue}
\newacronym{gju}{GJU}{Galileo Joint Undertaking}
\newacronym{gcc}{GCC}{Ground Control Center}
\newacronym{gmc}{GMC}{Ground Mission Center}
\newacronym{gcs}{GCS}{Galileo Control System}
\newacronym{gms}{GMS}{Galileo Mission System}
\newacronym{ttc}{TT\&C}{Telemetry Tracking and Control}
\newacronym{gss}{GSS}{Galileo Sensor Station}
\newacronym{uls}{ULS}{Up-Link Stations}
\newacronym{odts}{OD\&TS}{Orbitography Determination and Time Synchronisation}
\newacronym{mboc}{MBOC}{Multiplexed Binary Offset Carrier}
\newacronym{boc}{BOC}{Binary Offset Carrier}
\newacronym{bpsk}{BPSK}{Binary Phase-Shift Keying}
\newacronym{os}{OS}{Open Service}
\newacronym{prs}{PRS}{Public Regulated Service}
\newacronym{cs}{CS}{Commercial Service}
\newacronym{sol}{SoL}{Safety of Life}
\newacronym{gtrf}{GTRF}{Galileo Terrestrial Reference Frame}
\newacronym{sw}{SW}{Synchronisation Word}
\newacronym{fec}{FEC}{Forward Error Correction}
\newacronym{crc}{CRC}{Cyclic Redundancy Check}
\newacronym{GTRF}{GTRF}{Galileo Terrestrial Reference Frame}

%GNSS glossary
\newglossaryentry{lfsr}{%
	name={Linear Feedback Shift Register},%
	description={Shift register whose input bit is a linear function of its previous state.
	The initial value of the LFSR is called the seed, and because the operation of the register is deterministic, the stream of values produced by the register is completely determined by its current (or previous) state. Likewise, because the register has a finite number of possible states, it must eventually enter a repeating cycle. However, an LFSR with a well-chosen feedback function can produce a sequence of bits which appears random and which has a very long cycle.
	Applications of LFSRs include generating pseudo-random numbers, pseudo-noise sequences, fast digital counters, and whitening sequences. Both hardware and software implementations of LFSRs are common}%
}
\newglossaryentry{gold_code}{%
	name={Gold Code},%
	description={A Gold code is a type of binary sequence which have bounded small cross-correlations within a set. It is useful when multiple devices are broadcasting in the same frequency range. A set of Gold code sequences consists of $ 2n - 1 $ sequences each one with a period of $ 2n - 1 $.
	The exclusive or of two different Gold codes from the same set is another Gold code in some phase.
	Within a set of Gold codes about half of the codes are balanced, i.e. the number of ones and zeros differs by only one}%
}
\newglossaryentry{BPSK}{%
	name={BPSK},%
	description={It is the simplest Frequency Modulation (FSK) in which digital information is transmitted through discrete frequency changes of a carrier signal.}%
}
\newglossaryentry{BOC}{%
	name={BOC},%
	description={It is a variation of \gls{QPSK} modulation in phase and in quadrature, whose name referred to Binary Offset Carrier.}%
}
\newglossaryentry{pseudolite}{%
	name={pseudolite},%
	description={The pseudolites, which are commonly static ground-based transmitters, emit GNSS-like signals, using similar GNSS ranging codes and carrier frequencies. The data message is either similar to the one of the respective GNSS signal or changed in order to transmit augmentation information of all other GNSS signals. The data rate is increased accordingly up to $1000\ bps$ and more.\\
	The pseudolite signals are not affected by ionospheric delay errors, and also tropospheric influences are reduced. The static position of the pseudolites allows to minimize the orbital error by calibration. The small distances between pseudolites and user receivers will result in fast geometry changes in kinematic applications. This favours carrier phase ambiguity resolution techniques. Apart from these advantages, the pseudolite concept has to deal with different power interference problems, multipath effects and time synchronization}%
}
\newglossaryentry{IRNSS}{%
	name={IRNSS},%
	description={Indian Regional Navigation Satellites System which provides an autonomous navigation for the Indian subcontinent. The space segment of the system consists of seven satellites, three geostationary and four located in geosynchronous orbits with an inclination angle of $29^\circ$. The ground segment consist of two master control stations and about $20$ monitoring stations. It provides dual-frequency service using the L-band in coallocation with \gls{GPS} L5 and Galileo E5A, and the S-frequency band. It provides a position accuracy of $10\ m$ over India.}%
}
%Orbital Parameters
\newglossaryentry{ecc}{%
	name={Eccentricity},%
	description={Parameter of the ellipse, describing how much it is elongated compared to a circle},%
	symbol={\ensuremath{e}}%
}
\newglossaryentry{semimajor_axis}{%
	name={Semi-major Axis},%
	description={Sum of the periapsis and apoapsis distances divided by two. For circular orbits, the semimajor axis is the distance between the centers of the bodies, not the distance of the bodies from the center of mass},%
	symbol={\ensuremath{a}}%
}
\newglossaryentry{inclination}{%
	name={Inclination},%
	description={Vertical tilt of the ellipse with respect to the reference plane, measured at the ascending node (where the orbit passes upward through the reference plane)},%
	symbol={\ensuremath{i}}%
}
\newglossaryentry{right_ascension}{%
	name={Right Ascension of the Ascending Node},%
	description={Horizontally orients the ascending node of the ellipse (where the orbit passes upward through the reference plane) with respect to the reference frame's vernal point},%
	symbol={\ensuremath{\Omega}}%
}
\newglossaryentry{perigee}{%
	name={Argument of Perigee},%
	description={Orientation of the ellipse in the orbital plane, as an angle measured from the ascending node to the periapsis (the closest point the second body comes to the first during an orbit)},%
	symbol={\ensuremath{\omega}}%
}
\newglossaryentry{mean_anomaly}{%
	name={Mean Anomaly},%
	description={Position of the orbiting body along the ellipse at a specific time},%
	symbol={\ensuremath{M_0}}%
}
\newglossaryentry{central_longitude}{%
	name={Central Longitude of Ground Trace},%
	description={It is the centre of the $8$-figure ground trace, and it is the centre of two longitudes of ascending and descending. The range of the central longitude might exceed in order to improve the availability with the high elevation property of satellites.},%
	symbol={\ensuremath{l}}%
}

%GPS GLOSSARY
\newglossaryentry{trilateration}{%
	name={Trilateration},%
	description={In geometry, it is the process of determining absolute or relative locations of points by measurement of distances, using the geometry of circles, spheres or triangles.In contrast to triangulation, it does not involve the measurement of angles.}%
}
\newacronym{ECEF}{ECEF}{\textit{Earth-Centered, Earth-Fixed}}%

\newglossaryentry{WGS84}{%
	name={WGS84},%
	description={\textbf{World Geodetic System 1984}. Es un sistema geodésico mundial que permite calcular cualquier punto en la Tierra utilizando tres unidades. Consiste en un patrón matemático de tres dimensiones que representa la Tierra similar a un elipsoide}%}%
}
\newglossaryentry{GDOP}{%
	name={GDOP},%
	description={It relates the distance measurement errors with the dispersion of the positioning measure.}%
	\newacronym{GDOP}{GDOP}{Geometric Dilution Of Precision}%
}

%GLONASS GLOSSARY
\newglossaryentry{FDMA}{%
	name={FDMA},%
	description={}%
	\newacronym{FDMA}{FDMA}{Frequency Division Multiple Access}
}
\newacronym{PZ-90}{PZ-90}{Parametry Zemli 1990}

%TACAN
\newacronym{TACAN}{TACAN}{\textit{Tactical Air Navigation System}}

% ADVANCED POSITIONING TECHNIQUES
\newacronym{RTK}{RTK}{Real-Time Kinematic}
\newacronym{PPP}{PPP}{Precise Point Positioning}
\newacronym{cors}{CORS}{Continuously Operating Reference Stations}
\newacronym{OTF}{OTF}{on the fly}

% NON-GNSS NAVIGATION
\newacronym{SoOP}{SoOP}{Signals of Opportunity}
% The following definitions will go in the list of symbols



%DELIVERABLE 1.4
\newacronym{NMR}{NMR}{\textit{Nuclear Magnetic Resonnance}}
\newacronym{ESG}{ESG}{\textit{Electro Static Gyroscope}}
\newacronym{FOG}{FOG}{\textit{Fiber Optic Gyroscope}}
\newacronym{RLG}{RLG}{\textit{Ring Laser Gyroscope}}
\newacronym{MEMS}{MEMS}{\textit{Micro Electro Mechanical Systems}}
\newacronym{SAW}{SAW}{Surface Acoustic Wave}
\newacronym{EKF}{EKF}{Extended Kalman Filter}
\newacronym{LiDAR}{LiDAR}{Light Detection and Ranging}
\newacronym{NOAA}{NOAA}{National Oceanic and Atmospheric Administration}
\newacronym{usa}{US}{United States}
\newacronym{ROS}{ROS}{Robot Operating System}
\newacronym{IMLE}{IMLE}{Iterative Maximum Likelihood Estimation}
\newacronym{IMU}{IMU}{\textit{Inertial Measurement Unit}}
\newacronym{MAV}{MAV}{\textit{Micro Air Vehicle}}
\newacronym{FM}{FM}{Frequency Modulation}
\newacronym{AM}{AM}{Amplitude Modulation}
\newacronym{EIRP}{EIRP}{Effective Isotropic Radiated Power}
\newacronym{SLAM}{SLAM}{\textit{Simultaneous Localization And Mapping}}
\newacronym{WLAN}{WLAN}{Wireless Local Area Networks}
\newacronym{WPAN}{WPAN}{Wireless Personal Area Networks}
\newacronym{AHRS}{AHRS}{Attitude and Heading Reference System}
\newacronym{ISA}{ISA}{International Standard Atmosphere}
\newacronym{TIR}{TIR}{Thermal InfraRed}
\newacronym{DEM}{DEM}{Digital Elevation Map}
\newacronym{ANS}{ANS}{Autonomous Navigation System}
\newacronym{FOV}{FOV}{\textit{Field Of Vision}}
\newacronym{ARW}{ARW}{Angle Random Walk}
\newacronym{MOEMS}{MOEMS}{Micro-Opto-Electro-Mechanical System}
\newacronym{SQUID}{SQUID}{Superconducting Quantum Interference Device}
\newacronym{AMR}{AMR}{Anisotropic Magnetoresistance}
\newacronym{GMR}{GMR}{Giant Magnetoresistance}
\newacronym{GMI}{GMI}{Giant Magnetoimpedance}
%\newglossaryentry{ohm}{type=symbols,name=ohm,
%symbol={\ensuremath{\Omega}},
%description=unit of electrical resistance}

%\newglossaryentry{angstrom}{type=symbols,name={\aa}ngstr\"om,
%symbol={\AA},sort=angstrom,
%description={non-SI unit of length}}
\newglossaryentry{CCD}{%
	name={Charge-coupled device},%
	description={The CCD is a technology in digital imaging. In a CCD image sensor, pixels are represented by p-doped MOS capacitors. These capacitors are biased above the threshold for inversion when image acquisition begins, allowing the conversion of incoming photons into electron charges at the semiconductor-oxide interface. The CCD is then used to read out these charges.}%
}
\newglossaryentry{PTU}{%
	name={Pan Tilt Unit},%
	description={Precise real-time pointing of any payload}%
}

%DELIVERABLE 2.1
\newacronym{SoW}{SoW}{Statement of Work}
\newacronym{IEEE}{IEEE}{Institute of Electrical and Electronic Engineers}
\newacronym{SoC}{SoC}{\textit{System on Chip}}
\newacronym{GPP}{GPP}{General Purpose Processor}
\newacronym{SDK}{SDK}{Software Development Kit}
\newacronym{GPL}{GPL}{General Public License}
\newacronym{MIMO}{MIMO}{Multiple Input Multiple Output}
\newacronym{AGC}{AGC}{Automatic Gain Control}
\newacronym{RX}{RX}{Receiver}
\newacronym{TX}{TX}{Transmitter}
\newacronym{FIR}{FIR}{Finite Impulse Response}
\newacronym{RMS}{RMS}{Root Mean Square}
\newglossaryentry{EVM}{%
	name={EVM},%
	description={\textbf{Error Vector Magnitude} is a measure used to quantify the performance of a digital radio transmitter or receiver. A signal sent by an ideal transmitter or received by a receiver would have all constellation points precisely at the ideal locations, however various imperfections in the implementation (such as carrier leakage, low image rejection ratio, phase noise etc.) cause the actual constellation points to deviate from the ideal locations. Informally, \acrshort{EVM} is a measure of how far the points are from the ideal locations. Represented by the equation:
	\begin{equation}
	EVM(dB)=10\log_{10} (\frac{P_{error}}{P_{reference}})
	\end{equation}
	where $P_{error}$ is the \acrshort{RMS} power of the error vector. For single carrier modulations, $P_{reference}$ is, by convention, the power of the outermost (highest power) point in the reference signal constellation. More recently, for multi-carrier modulations, $P_{reference}$ is defined as the reference constellation average power.}%
	\newacronym{EVM}{EVM}{Error Vector Magnitude}%
}
\newacronym{FDD}{FDD}{Frequency Division Duplex}
\newacronym{PLL}{PLL}{Phase-Locked Loop}
\newacronym{PA}{PA}{Power Amplifier}
\newacronym{OE}{OE}{OpenEmbedded}
\newglossaryentry{BitBake}{%
	name={BitBake}, %
	description={It is a generic task executor and scheduler used by the \gls{OE} build system to build images, allowing shell and Python tasks to be run efficiently and in parallel while working within complex inter-task dependency constraints.}
}
\newacronym{VCO}{VCO}{Voltage Controlled Oscillator}
\newacronym{IC}{IC}{Integrated Circuit}
\newacronym{CSP}{CSP}{Chip Scale Package}
\newacronym{BGA}{BGA}{Ball Grid Array}
\newglossaryentry{OpenedHand}{ %
	name={OpenedHand}, %
	description={It was an embedded Linux start-up that was acquired by Intel. The firm developed an \gls{OE} distribution called Poky Linux,now part of the Yocto Project. and the Clutter library.}}
\newglossaryentry{Sensitivity}{ %
	name={Sensitivity}, %
	description={The smallest signal that a network can reliably detect. Sensitivity specifies the strength of the smallest signal	at the input of a network that causes the output signal power 
	to be $M$ times the output noise power where $M$ must be specified. $M=1$ is very popular. For a source temperature of $290\ K$, the relationship of sensitivity to noise figure is in dBm:
	\begin{equation}
	\begin{aligned}
	S_i&= MkT_0BF \\
	S_i(dBm)&= -174\ dBm + F(dB) + 10\log_{10}B +10\log_{10}M
	\end{aligned}
	\end{equation}}
	where $k$ is Boltzmann Constant; $T_0$ is the temperature of the receiver ($290\ K$); $B$ is the receiver bandwidth and $F$ the Noise Figure}
\newacronym{UHD}{UHD}{USRP Hardware Drive}
\newacronym{AL}{AL}{Abstraction Layer}
\newglossaryentry{API}{ %
	name={API}, %
	description={\textbf{Application Programming Interface}. Being \gls{API} the set of subroutines, functions and procedures which offer the possibility of using a type of library by another software as an \gls{AL}, it represents the communication capability between software or hardware.} %
	\newacronym{API}{API}{Application Programming Interface} %
}
\newacronym{PS}{PS}{Processing System}
\newacronym{PL}{PL}{Programmable Logic}
\newacronym{CPU}{CPU}{Central Processor Unit}
\newglossaryentry{ASSP}{%
	name={ASSP},%
	description={\textbf{Application-Specific Standard Product} is a semiconductor device  \gls{IC} product that is dedicated to a specific application market and sold to more than one user (and thus, "standard"). The \gls{ASSP} is marketed to multiple customers just as a general-purpose product is, but to a smaller number of customers since it is for a specific application. Like an \gls{ASIC}, the \gls{ASSP} is for a special application, but it is sold to any number of companies.}%
	\newacronym{ASSP}{ASSP}{Application-Specific Standard Product}%
}
\newacronym{ASIC}{ASIC}{Application-Specific Integrated Circuit}
\newacronym{MMU}{MMU}{Memory Management unit}
\newacronym{DMA}{DMA}{Direct Memory Access}
\newacronym{GIC}{GIC}{General Interrupt Controller}
\newacronym{WDT}{WDT}{Watch Dog Timers}
\newacronym{ttc2}{TTC}{Two Triple Timers/Counters}
\newacronym{PTM}{PTM}{Program Trace Macrocell}
\newacronym{CTI}{CTI}{Cross Trigger Interface}
\newacronym{DDR}{DDR}{Double Data Rate}
\newacronym{SDRAM}{SDRAM}{Synchronous Dynamic Random-Access Memory}
\newacronym{LPDDR}{LPDDR}{Low Power Double Data Rate}

\newacronym{SPI}{SPI}{\textit{Serial Peripheral Interface}}%
\newglossaryentry{pps}{ %
	name={PPS}, %
	description={\textbf{Pulse Per Second} is an electrical signal that has a width of less than one second and a sharply rising or abruptly falling edge that accurately repeats once per second.} %
	\newacronym{pps}{pps}{Pulse Per Second}%
}
\newglossaryentry{UART}{ %
	name={UART}, %
	description={It is the protocol that specifies the electrical features for the communication in an interface.}
	\newacronym{UART}{UART}{Universal Asynchronous Receiver/Transmitter}%
}
\newacronym{USB}{USB}{\textit{Universal Serial Bus}} %

\newacronym{I/O}{I/O}{Input/Output}
\newglossaryentry{SMB}{ %
	name={SMB}, %
	description={\textbf{SubMiniature version B} is a coaxial \gls{RF} connector} %
	\newacronym{SMB}{SMB}{SubMiniature version B} %
}
\newglossaryentry{GPIO}{ %
	name={GPIO}, %
	description={\textbf{General Purpose Input/Output} is a generic pin on an integrated circuit whose behaviour is controllable by the user at run time.} %
	\newacronym{GPIO}{GPIO}{General Purpose Input/Output}
}
\newacronym{ATR}{ATR}{Automatic Transmit/Receive}
\newacronym{TCK}{TCK}{Test Clock}
\newacronym{TMS}{TMS}{Test Mode Select}
\newacronym{TDI}{TDI}{Test Data In}
\newacronym{TDO}{TDO}{Test Data Out}
\newacronym{JTAG}{JTAG}{Joint Test Action Group}
\newacronym{SDIO}{SDIO}{Secure Digital Input Output}
\newacronym{SD}{SD}{Secure Digital}

\newacronym{APU}{APU}{Application Processor Unit}
\newacronym{QoS}{QoS}{Quality of Service}
\newacronym{CLB}{CLB}{Configurable Logic Block}
\newacronym{LUT}{LUT}{LookUp Tables}
\newacronym{RAM}{RAM}{Random Access Memory}
\newacronym{LVCMOS}{LVCMOS}{Low Voltage Complementary Metal Oxide Semiconductor}
\newacronym{LVDS}{LVDS}{Low-Voltage Differential Signaling}
\newacronym{SSTL}{SSTL}{Stub Series Terminated Logic}
\newacronym{AXI}{AXI}{Advanced eXtensible Interface}
\newacronym{ACP}{ACP}{Accelerator Coherency Port}
\newacronym{OCM}{OCM}{On-Chip Memory}
\newacronym{FIFO}{FIFO}{First Input First Output}
\newacronym{SCU}{SCU}{Snoop Control Unit}
\newacronym{IP}{IP}{Internet Protocol}
\newglossaryentry{DHCP}{ %
	name={DHCP}, %
	description={\textbf{Dynamic Host Configuration Protocol} is a standardized network protocol used on \gls{IP} networks for dynamically distributing network configuration parameters, such as \gls{IP} addresses for interfaces and services. Computers request \gls{IP} addresses and networking parameters automatically from a DHCP server, reducing the need for a network administrator or a manual configuration from a user.} %
	\newacronym{DHCP}{DHCP}{Dynamic Host Configuration Protocol} %
}
\newglossaryentry{GNU}{ %
	name={GNU}, %
	description={It is a gls{UNIX}-like \gls{OS} which consists in a free software that allow users to run, copy, distribute, study, change and improve the software. It has a collection of many programs, applications, libraries and developer tools. In a combination with a Linux kernel for its use, it is called GNU/Linux \gls{OS}.}
	\newacronym{GNU}{GNU}{GNU} %
}
\newglossaryentry{UNIX}{ %
	name={UNIX},
	description={It is a family of multitasking, multiuser computer \gls{OS} that derive from the original AT\&T Unix.}
	\newacronym{UNIX}{UNIX}{UNIX} %
}
\newglossaryentry{VIVADO}{ %
	name={Xilinx Vivado},
	description={It is a design environment for \gls{FPGA} products from Xilinx. It enables developers to synthesize their designs, perform timing analysis, examine RTL diagrams, simulate a design's reaction to different stimuli, and configure the target device with the programmer.}
	\newacronym{VIVADO}{VIVADO}{Xilinx Vivado} %
}

% TFG Marta
\newacronym{ACAS}{ACAS}{Airborne Collision Avoidance System}
\newacronym{ADSB}{ADS-B}{Automatic Dependent Surveillance-Broadcast}
\newacronym{ATC}{ATC}{Air Traffic Control}
\newacronym{ATCRBS}{ATCRBS}{Air Traffic Control Radar Beacon System}
\newacronym{BCAS}{BCAS}{Beacon Collision Avoidance System}
\newacronym{CLD}{CLD}{Color Layout Descriptor}
\newacronym{CSD}{CSD}{Color Structure Descriptor}
\newacronym{csd}{CSD}{Contour-based Shape Descriptor}
\newacronym{DCD}{DCD}{Dominant Color Descriptor}

\newacronym{EHD}{EHD}{Edge Histogram Descriptor}
%\newacronym{FAA}{FAA}{Federal Aviation Administration}
\newacronym{GA}{GA}{General Aviation}
\newacronym{GoF}{GoF}{Group of Frame}
\newacronym{HTD}{HTD}{Homogeneous Texture Descriptor}
%\newacronym{ICAO}{ICAO}{International Civil Aviation Organization}
\newacronym{MCTOM}{MCTOM}{Maximum Certified Take-Off Mass}
\newacronym{MTOW}{MTOW}{Maximum Take-Off Weigth}
\newacronym{NextGen}{NextGen}{Next Generation Air Transportation System}
\newacronym{pa}{PA}{Proximity Advisories}
\newacronym{QHDM}{QHDM}{Quadratic Histogram Distance Measure}
\newacronym{RA}{RA}{Resolution Advisories}
\newacronym{RGB}{RGB}{\textbf{Red Green Blue}}
\newacronym{RSD}{RSD}{Region-based Shape Descriptor}

\newacronym{SCD}{SCD}{Scalable Color Descriptor}
\newacronym{SESAR}{SESAR}{Single European Sky Research}
\newacronym{ATM}{ATM}{Air Traffic Management}
%\newacronym{SNR}{SNR}{Signal to Noise Ratio}
\newacronym{SSR}{SSR}{Secondary Surveillance Radar}
\newacronym{SVO}{SVO}{Selective Velocity Obstacle}
\newacronym{TA}{TA}{Traffic Advisories}
\newacronym{TBD}{TBD}{Texture Browsing Descriptor}
\newacronym{TCAS}{TCAS}{Traffic Alert and Collision Avoidance System}
\newacronym{UAV}{UAV}{\textit{Unmanned Aerial Vehicle}}
\newacronym{UAS}{UAS}{Unmanned Aerial System}
\newacronym{UCAV}{UCAV}{Unmanned Combat Aerial System}
\newacronym{VO}{VO}{Velocity Obstacle}


\newglossaryentry{LLOYD}{%
	%
name={Algoritmo de Lloyd Generalizado},
description={También llamado algoritmo K-means, se trata de un algoritmo de método de agrupamiento, que tiene como objetivo la partición de un conjunto de n observaciones en k grupos en el que cada observación pertenece al grupo cuyo valor medio es más cercano. Utiliza una técnica de refinamiento iterativo}%
%
}

\newglossaryentry{GABOR}{%
	%
	name={Funciones de Gabor},
	description={Se trata de funciones que están presentes tanto en el dominio espacial como en el de la frecuencia. Es un paquete de ondas gaussiano, es decir, con una envolvente gaussiana que la localiza espacialmente. Fueron propuestas como un método óptimo de descomponer una señal en cuantos de información}%
	%
}

\newglossaryentry{DCT}{%
	%
	name={Transformada de Coseno Discreta},
	description={Expresa una secuencia finita de varios puntos como resultado de la suma de distintas señales sinusoidales (con distintas frecuencias y amplitudes). Trabaja con una serie de números finitos y sólo con cosenos. Es una función lineal e invertible del dominio real $R_N$ al dominio real $R_N$}%
	%
	\newacronym{DCT}{DCT}{Discrete Cosine Transform}%
	%
}

\newglossaryentry{RVSM}{%
	%
	name={RVSM},
	description={Término usado en aviación para la separación vertical mínima requerida entre dos aeronaves. Mediante la introducción de este concepto se redujo la separación vertical de 2000 ft a 1000ft para niveles de vuelo por encima de FL290. Con ello se consiguió aumentar los niveles de vuelo disponibles. Además, las aeronaves pueden volar más cerca de su nivel de vuelo óptimo, lo que supone un ahorro de combustible}%
	\newacronym{RVSM}{RVSM}{Reduced Vertical Separation Minima}%
	%
}

\newacronym{SES}{SES}{Single European Sky}
\newacronym{TFG}{TFG}{Trabajo Fin de Grado}

% TFG IFFINAZZI
\newacronym{FHD}{FHD}{\textit{Full High Definition}}
\newacronym{MCUimg}{MCU}{\textit{Minimum Coded Unit}}
\newacronym{DCTEnglish}{DCT}{\textit{Discrete Cosine Transform}}
\newacronym{RT}{RT}{\textit{Real Time}}
\newacronym{DIY}{DIY}{\textit{Do It Yourself}}
\newacronym{DoF}{DoF}{\textit{Degrees of Freedom}}
\newacronym{PCB}{PCB}{\textit{Printed Circuit Board}}
\newacronym{MIT}{MIT}{\textit{Massachusetts Institute of Technology}}
\newacronym{CR}{CR}{\textit{Carrier Return}}
\newacronym{LF}{LF}{\textit{Line Feed}}
\newacronym{CMY}{CMY}{\textit{Cyan Magenta Yellow}}
\newacronym{HSL}{HSL}{\textit{Hue Saturation Lightness}}
\newacronym{HSI}{HSI}{\textit{Hue Saturation Intensity}}
\newacronym{ccd}{CCD}{\textit{Charge-Coupled Device}}
\newacronym{CMOS}{CMOS}{\textit{Complementary Metal-Oxide Semiconductor}}
\newacronym{ddp}{ddp}{Diferencia de Potencial}
\newacronym{MOS}{MOS}{\textit{Metal-Oxide Semiconductor}}
\newacronym{aps}{APS}{\textit{Active Pixel Sensor}}
\newacronym{ecef}{ECEF}{\textit{Earth Centered Earth Fixed}}
\newacronym{LLS}{LLS}{\textit{Local Level System}}
\newacronym{BFS}{BFS}{\textit{Body Fixed System}}
\newacronym{cdm}{CdM}{Centro de Masas}
\newacronym{LIDAR}{LIDAR}{\textit{Light Detection and Ranging}}
\newacronym{RANSAC}{RANSAC}{\textit{RANdom SAmple Consensus}}
\newacronym{SVM}{SVM}{\textit{Support Vector Machine}}
\newacronym{MLDA}{MLDA}{\textit{Multiresolution Linear Discriminant Analysis}}

\newglossaryentry{scriptMATLAB}{%
	%
	name={Script de Matlab},
	description={Es un programa que se almacena en un archivo de texto plano. Al ser escrito en lenguaje Matlab, las directrices que éste contiene son interpretadas y para poder ejecutarse se necesita Matlab, el intérprete del script}%
	%
}

\newglossaryentry{frame}{%
	%
	name={Frame},
	description={Es cada una de las imágenes que conforman un vídeo. Se usa comúnmnete en el ámbito de la fotografía digital como parámetro del sensor para indicar el número de imágenes por segundo que éste es capaz de capturar. $\frac{frames}{sec}=\frac{imágenes}{segundo}$}%
	%
}

\newglossaryentry{fov}{%
	%
	name={Campo de Visión},
	description={Del inglés \textit{Field of Vision}. Es el ángulo de apertura que tiene la cámara. Relacionado con la resolución de la cámara se puede obtener una densidad de píxeles por metro -o unidad de medida de distancia- para una distancia dada a un objeto de la imagen. A mayor densidad de $\frac{píxeles}{u.de~longitud}$ mejor se verán los detalles de un objeto en una imagen.}%
	%
	\newacronym{fov}{FoV}{Field of View}
}

\newglossaryentry{gimbal}{%
	%
	name={Gimbal},
	description={Es un sistema de estabilización utilizado en la filmación en movimiento. Está compuesto por tres ejes de giro perpendiculares entre sí, tres servomotores ligados a cada eje de giro y una \gls{IMU} que calcula los ángulos de giro de la plataforma. El funcionamiento consiste en que la \gls{IMU} reporta a cada servomotor de que ángulo debe corregir para que, ante los giros de la plataforma, esta se mantenga estable }%
	%
}

\newglossaryentry{VFW}{%
	%
	name={Video For Windows},
	description={Driver para cámaras con numerosos parámetros de configuración para los sistemas operativos Windows}%
	%
	\newacronym{VFW}{VFW}{Video For Windows}
}

\newglossaryentry{V4L2}{%
	%
	name={Video For Linux 2},
	description={Driver para cámaras con numerosos parámetros de configuración para los sistemas operativos Linux}%
	%
	\newacronym{V4L2}{V4L2}{Video For Linux 2}
}

\newacronym{I2C}{I2C}{\textit{Inter-Integrated Circuit}}

\newacronym{SPIesp}{SPI}{\textit{Serial Peripheral Interface}}%
\newacronym{RLE}{RLE}{\textit{Run-Length Enconding}}%